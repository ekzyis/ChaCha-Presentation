\documentclass{beamer}
%
% Choose how your presentation looks.
%
% For more themes, color themes and font themes, see:
% http://deic.uab.es/~iblanes/beamer_gallery/index_by_theme.html
%
\mode<presentation>
{
  \usetheme{Warsaw}      % or try Darmstadt, Madrid, Warsaw, ...
  \usecolortheme{beaver} % or try albatross, beaver, crane, ...
  \usefonttheme{default}  % or try serif, structurebold, ...
  \setbeamertemplate{navigation symbols}{}
  \setbeamertemplate{caption}[numbered]
}

\usepackage[english]{babel}
\usepackage[utf8x]{inputenc}

\title[Impl. and Visualization of the ChaCha Cipher Family in CT2]{Implementation and Didactical Visualization of the ChaCha Cipher Family in CrypTool 2}
\author{Ramdip Gill}
\institute{University of Heidelberg}
\date{January 28, 2021}

\begin{document}

\begin{frame}[plain]
  \titlepage
\end{frame}

% Uncomment these lines for an automatically generated outline.
\begin{frame}[plain]{Outline}
  \tableofcontents
\end{frame}

\section{Einleitung}

\subsection{Was ist CrypTool 2?}
\begin{frame}{Was ist CrypTool 2?}
\end{frame}

\subsection{Warum ChaCha?}
\begin{frame}{Warum ChaCha?}
\end{frame}

\section{Ziele des Plug-ins}
\begin{frame}{Ziele des Plug-ins}
\end{frame}

\section{ChaCha Spezifikation}
\begin{frame}{ChaCha Spezifikation}
\end{frame}

\subsection{Quarterround-Funktion}
\begin{frame}{Quarterround-Funktion}
\end{frame}

\subsection{Little-Endian-Funktion}
\begin{frame}{Little-Endian-Funktion}
\end{frame}

\subsection{Hashfunktion}
\begin{frame}{Hashfunktion}
\end{frame}

\subsection{Aufbau der Zustandsmatrix}
\begin{frame}{Aufbau der Zustandsmatrix}
\end{frame}

\section{Live-Präsentation des Plug-ins}
\begin{frame}{Live-Präsentation des Plug-ins}
\end{frame}

\section{Architektur}
\begin{frame}{Architektur}
\end{frame}


\end{document}